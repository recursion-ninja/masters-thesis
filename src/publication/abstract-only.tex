\documentclass[runningheads]{llncs}
%
\usepackage{graphicx}
% Used for displaying a sample figure. If possible, figure files should
% be included in EPS format.
%
% If you use the hyperref package, please uncomment the following line
% to display URLs in blue roman font according to Springer's eBook style:
% \renewcommand\UrlFont{\color{blue}\rmfamily}

% --- --- --- --- --- --- --- --- --- --- --- --- --- --- --- --- --- --- --- ---
% --- --- --- Setup glossary options for abbreviations
\usepackage[acronyms,shortcuts,toc,section=chapter,numberedsection=autolabel]{glossaries}
\setacronymstyle{long-short}
\makenoidxglossaries
\newcommand{\Abrev}[1]{\gls{#1}}
\newcommand{\DefineAbbreviation}[2]{\newacronym{#1}{#1}{#2}}
\newcommand{\OutputAbreviations}{%
	\printnoidxglossary[type=acronym,title={Appendix: Abreviations}]
	\clearpage
}


% --- --- --- --- --- --- --- --- --- --- --- --- --- --- --- --- --- --- --- ---
% --- --- --- Define abbreviations used
\DefineAbbreviation{AEAD}{Authenticated Encryption with Associated Data}
\DefineAbbreviation{ART}{Asynchronous Ratcheting Tree}
\DefineAbbreviation{AS}{Authentication Service}
\DefineAbbreviation{BAN logic}{Burrows---Abadi---Needham logic}
\DefineAbbreviation{BDD}{Binary Decision Tree}
\DefineAbbreviation{CGKA}{Continuous Group Key Agreement}
\DefineAbbreviation{DS}{Delivery Service}
\DefineAbbreviation{FS}{(Perfect) Forward Secrecy}
\DefineAbbreviation{FSM}{Finite State Machine}
\DefineAbbreviation{FSU}{Forward Secrecy \emph{with Updates}}
\DefineAbbreviation{IETF}{Internet Engineering Task Force}
\DefineAbbreviation{ITM}{Internet threat model}
\DefineAbbreviation{LBBT}{left-balanced binary tree}
\DefineAbbreviation{LTS}{Labeled Transition System}
\DefineAbbreviation{LTL}{Linear Temporal Logic}
\DefineAbbreviation{MLS}{Message Layer Security}
\DefineAbbreviation{PCS}{Post-compromise Security}
\DefineAbbreviation{SM}{Secure Messaging}
\DefineAbbreviation{SGM}{Secure Group Messaging}


\usepackage{adjustbox}
\usepackage{amsmath}
\usepackage{amssymb}
\usepackage{booktabs}
\usepackage{copyrightbox} % for copyrighted figures
\usepackage{etoolbox}
\usepackage{graphicx}
\usepackage{import}
\usepackage{makecell} % for \thead and multi-ine header cells
\usepackage{pifont}
\usepackage[indentfirst=false]{quoting}
\usepackage{stmaryrd} % double square bracket / parens
\usepackage{xfrac} % for sfrac in \NA
\usepackage{xspace}

% Shorthands for: CGKA 
\newcommand{\CGKAdef}{\ensuremath{{}^{\ulcorner}\!\!\textbf{\textit{CGKA}}_{\!\lrcorner}}\xspace}%
\newcommand{\CGKAsec}{\Abrev{CGKA} security game\xspace}%
\newcommand{\CGKAmod}[3]{\ensuremath{\mathcal{M}_{\texttt{CGKA}}%
		\expandafter\ifstrempty\expandafter{#1}
		{
			\expandafter\ifstrempty\expandafter{#2}
			{
				\expandafter\ifstrempty\expandafter{#3}
				{}
				{\langle\,#3\,\rangle}
			}
			{
				\expandafter\ifstrempty\expandafter{#3}
				{\langle\,#2\,\rangle}
				{\langle\,#2,\, #3\,\rangle}
			}
		}
		{
			\expandafter\ifstrempty\expandafter{#2}
			{
				\expandafter\ifstrempty\expandafter{#3}
				{\langle\,\mathtt{#1}\,\rangle}
				{\langle\,\mathtt{#1},\, #3\,\rangle}
			}
			{
				\expandafter\ifstrempty\expandafter{#3}
				{\langle\,\mathtt{#1},\, #2\,\rangle}
				{\langle\,\mathtt{#1},\, #2,\, #3\,\rangle}
			}
		}
	}%
	\xspace
}%

% Shorthands for: TreeKEM versions
\newcommand{\VersionAny}[1]{\ensuremath{\mathtt{v#1}^{\mathtt{\textbf{\scalebox{.55}[1.0]{TreeKEM}}}}}\xspace}%
\newcommand{\VersionOne}{\VersionAny{1}}%
\newcommand{\VersionTwo}{\VersionAny{2}}%

% Shorthands for: Oracle and Protocol references
\newcommand{\Protocol}[1]{\texttt{\textit{#1}}\xspace}
\newcommand{\Oracle}[2]{%
	{\ttfamily%
		\textbf{#1}%
		\expandafter\ifstrempty\expandafter{#2}{()}%
		{(\(\,\mathtt{#2}\,\))}%
	}\xspace%
}%


\begin{document}
	%
	\title{Verifying the TreeKEM protocol with SPIN and Continuous Group Key Agreement}
	%
	%\titlerunning{Abbreviated paper title}
	% If the paper title is too long for the running head, you can set
	% an abbreviated paper title here
	%
	\author{Alex J. Washburn\inst{1,2}\orcidID{0000-0001-7181-4288} \and\\
		Subash Shankar\inst{1,3}}
	%
	\authorrunning{F. Author et al.}
	% First names are abbreviated in the running head.
	% If there are more than two authors, 'et al.' is used.
	%
	
	\institute{Hunter College, New York NY  10065, USA\\
		\url{https://hunter.cuny.edu/csci} \and
		\email{academia@recursion.ninja} \and
		\email{subash.shankar@hunter.cuny.edu}}
	%
	\maketitle              % typeset the header of the contribution
	%
	\begin{abstract}
		The TreeKEM protocol is the preeminent implementation candidate for the Message Layer Security standard.
		Prior work analyzed TreeKEM by defining the Continuous Group Key Agreement security game,
		which facilitated proof of some security guarantees as well as identifying protocol deficiencies which were subsequently remedied.
		This work extends the such applications by formalizing the Continuous Group Key Agreement security game through multiple  soundness preserving abstractions.
		Once formalized, the game is encoded it within Promela and security guarantees of TreeKEM are formally verifying via the explicit state model checker Spin.
		The model is parameterized by $N$, representing an infinite TreeKEM protocol execution whose group membership not exceed $N$ unique individuals.
		Emphasis is placed on security game formalization through abstraction composition and scalability of the model encoding with respect the parameter $N$.
		Verification results shows that the security guarantees of ``Post-Compromise Security'' and ``Forward Secrecy with Updates'' hold for TreeKEM group sizes $\le 24$.
		These findings are discussed within the context of other TreeKEM security analyses.
		This represents a notable achievement, both in practical security terms for Message Layer Security users and the TreeKEM protocol, as well as demonstrating scalability techniques for non-trivial parameter bounds when modeling a complex, concurrent protocol.
		
		\keywords{Cryptographic Protocols, Formal Verification, Linear Temporal Logic, Model Checking, Promela, Spin.}
	\end{abstract}
	
	
\end{document}
