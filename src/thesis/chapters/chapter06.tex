\hypertarget{sec:conclusion}{%
\chapter{Conclusion}\label{sec:conclusion}}

The main contribution of this work is the formalization of the \CGKAsec\ as well as presenting an accompanying parametric and protocol-polymorphic \CGKAmod{P}{}{} model.
At the time of writing, there are no other known formalizations or explicit modeling of the TreeKEM protocol, making this work an important step in the formal verification of TreeKEM.
The resulting \CGKAmod{}{}{} model incorporates a variety of sound reductions from the informal \CGKAsec\ definition, leading to dramatically simplifies model encoding.
The presented \CGKAmod{}{}{} model is capable of verifying TreeKEM for any parameters provided, limited only by the available hardware.
Performance tuning of the model along with performance observations are documented in detail.
Furthermore, scalability is scrutinized by estimating the run-time of increasingly large \(T\) and \(N\) parameters for the \CGKAmod{}{}{} model.
Regrettably, without further improvements to the model encoding, there is very little space for increasing the \(T\) and \(N\) parameter values.

\hypertarget{sec:future-work}{%
\section{Future Work}\label{sec:future-work}}

Publication of this thesis is far from the terminus of the work herein
Continued research building upon the contents of this thesis are numerous.


\hypertarget{sec:hypothesis-outcomes}{%
\section{Hypothesis Testing}\label{sec:hypothesis-outcomes}}

Accompanying the description of \Abrev{CGKA}, \CGKAdef, and the \CGKAsec\ the authors also include an application of mathematically analyzing TreeKEM with the \CGKAsec.
The result of the analysis was a constructive counter-example demonstrating that TreeKEM, as originally defined, is \emph{not} \Abrev{FS}.
Following the revelation of TreeKEM's deficiency the same authors present a new definition of TreeKEM, substituting the use of a standard PKE cryptographic primitive with UPKE, which remedies the defect and regains the \Abrev{FSU} security guarantee for TreeKEM.\@

Future work will explore verification of the both versions of TreeKEM.\@
Verification of TreeKEM will consider both the original definition of TreeKEM as well as the remedied version of TreeKEM, denoted by \VersionOne\ and \VersionTwo; respectively.
With the knowledge of \VersionOne possessing a \Abrev{FS} deficiency, this informs our expectation of the verification results.
These expectations can form a series of verification hypotheses which are depicted in Table\ \ref{tab:verification-hypotheses}.

\begin{table}[ht!]                                                                                                       \centering 
\caption{%
\label{tab:verification-hypotheses}%
Verification Hypotheses of \CGKAmod{P}{}{}.
}%
\subimport{../tables/}{verification-hypotheses}
\end{table}

With these verification hypotheses in mind, any verification outcome which falsifies a hypothesis is an interesting result.
If a counter-example is produced violating \Abrev{PCS} for either version of the protocol, then a previously unknown security defect for TreeKEM has been discovered.
If a counter-example is produced violating \Abrev{FSU} for the revised \VersionTwo, this also discovers a security defect.

The original \VersionOne\ has been shown to have a \Abrev{FSU} deficiency for a \((12, 12, 8)\)-attacker \(\mathcal{A}\) \autocite{alwen2020security}.
Hence a successful \Abrev{FSU} verification for the original \VersionOne\ with parameters corresponding to \(\mathcal{A}\) indicates either an error in the vulnerability provided by \autocite{alwen2020security} or, more likely, an error the TreeKEM model encoding.
While prior work \autocite{alwen2020security} indicates \( \CGKAmod{\VersionOne}{12}{8} \not \models \LTLPredicate{FSU} \), it is possible that a simpler model with lower \(T\) and/or \(N\) parameters may exist which also demonstrates the \Abrev{FSU} security deficiency.
Similarly, there may be many \CGKAmod{\VersionOne}{T}{N} models with simpler \(T\) and \(N\) parameters which verify \LTLPredicate{FSU}.
The aforementioned single \Abrev{FSU} security deficiency example point does not delineate a boundary at which \((T, C, N)\)-attacker does not demonstrate advantage but a \((T+1, C, N)\)-attacker or a \((T, C, N+1)\)-attacker is able to demonstrate advantage for the original protocol \VersionOne.
Locating the minimal \(T\), \(C\), and \(N\) values for which the vulnerability manifests in the \VersionOne version.
In order to ensure at least one observation of the \Abrev{FSU} deficiency of the original \VersionOne, the bounds are set to \(12 = \Tmax\) and \(8 = \Nmax\).
Verification observations related to \LTLPredicate{FSU} ought to be the most intriguing.

Completing this future work requires continued verification of different parameterizations of \CGKAmod{P}{T}{N} models in order to increase the tabulated observations.
Most importantly will be the observation of the model \CGKAmod{\VersionOne}{12}{8}, as this will determine the verification methodology's relationship with prior work and known security issues.
By interpreting the hypothetically complete observations collected in an expanded Appendix\ \ref{apx:Observations} one can discern whether the hypotheses enumerated in Table\ \ref{tab:verification-hypotheses} and described in Chapter\ \ref{sec:methodology} have been supported or falsified.
The author intends to pursue this work in the immediate future.


\hypertarget{sec:more-predicates}{%
\subsection{Adding LTL Predicates}\label{sec:more-predicates}}

The general construction of the \CGKAmod{P}{T}{N} lends itself well towards extension and repeat use.
Consideration and verification of other desirable properties for \CGKAdef\ protocols would be a logical extension of this work.
One particular area where the \CGKAmod{P}{T}{N} model may be particularly well suited, is the exploration of loosening the \Abrev{MLS} specification's constraints on the \Abrev{DS} semantics.
Perhaps fewer requirement can be placed on the \Abrev{DS}, required protocol to assume less from the \Abrev{DS}, while still maintaining all the \Abrev{MLS} security guarantees.
Pursuing this, or similar inquiries, would require minor modifications to the \CGKAmod{P}{T}{N} model's implementation.

Another avenue of consideration is permitting the adversary to be a group member for some portion of the \CGKAdef protocol.
After being removed from the group, can the adversary deny message confidentiality for remaining group members?
This would require carefully defining when and how the adversary enters and exits the communication group.
The permitted segment of adversary group membership would be different than corrupting a group member and consequently would effect the implication precondition of \Abrev{PCS} encoded as \Abrev{LTL} as in Section \ref{sec:pcs-as-ltl}.


\hypertarget{sec:unbounded-verification}{%
\subsection{Unbounded Verification}\label{sec:unbounded-verification}}

The final suggestion of future work was alluded to in Section\ \ref{sec:game-adaptations}, which involves identifying a finite model encoding for an unbounded \(T\) and/or \(N\) parameter.
Obviously the verification results would be stronger for a new \( \CGKAmod{P}{T}{N}^{'} \) model which considered \emph{all} possible epochs \(T\) or \emph{all} possible group sizes \(N\).
The dependence on parameter \(N\) for tracking group membership and the left balanced binary tree with the attacker's knowledge-base representation, leads the author to believe that a finite encoding with respect to the parameter \(N\) is likely impossible to achieve such an encoding for \(T\), both the global state variables of \( \CGKAmod{P}{T}{N}^{'} \) as well as the attacker knowledge-base module for \texttt{P} would require ``independence'' for from the epoch information.  
One possible approach may be finding an encoding of \( \CGKAmod{P}{T}{N}^{'} \) which is a Markov process \autocite{markov1906a}.
