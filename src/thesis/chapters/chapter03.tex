\hypertarget{sec:model-formalization}{%
\chapter{Model Formalization}\label{sec:model-formalization}}


\hypertarget{sec:game-adaptations}{%
\section{CGKA game formalization}\label{sec:game-adaptations}}

The verification results of a model have no bearing on TreeKEM unless the model is a faithful and equivalent translation of the protocol.
To verify the TreeKEM protocol, the \CGKAsec\ is used to derive the model for verification.
Proof that the TreeKEM protocol is a \Abrev{CGKA} protocol, as well as proof that results of the \CGKAsec\ for any \Abrev{CGKA} protocol are valid for \Abrev{FSU} and \Abrev{PCS}, have already been produced \autocite{alwen2020security}.
Formalizing and justifying the equivalence of this work's modeling of both the \CGKAsec\ and the side channel attacker knowledge gained while observing the TreeKEM protocol, is pivotal to the soundness of verifying \Abrev{FSU} and \Abrev{PCS} for TreeKEM.\@

Within this chapter we will formalize a parameterized model \CGKAmod{P}{T}{N}.
For a given LTL predicate $\varphi$ we define \CGKAmod{P}{T}{N} as follows:

\[
\begin{split}
\CGKAmod{P}{T}{N}\quad\models\quad& \varphi \;\;\;\textrm{ iff }\;\;\;\forall\; t \in [1, T],\, n \in [2, N],\, c \in [1, T]\\
  & \varphi \textrm{ holds in all possible \CGKAsec\!\!s}\\
  & \textrm{with a \CGKAdef\ compatible protocol } \mathtt{P}\\
  & \textrm{played by an } (t, c, n)\textrm{-attacker}
\end{split}
\]


Recall from Section \ref{sec:security-games} that group members can only call the six \CGKAdef\ algorithms, while the adversary can only query the ten oracles of the \CGKAsec.
The informal description of the \CGKAsec\ defined by \autocite{alwen2020security} places very few constraints on the queries an adversary can make.
These constraints are only related to the \(C\) parameter, limiting the total number of queries to \Oracle{reveal}{} and \Oracle{chall}{} to be less than or equal to \(C\).
Furthermore, the adversary is limited to making at most \emph{one} query to either \Oracle{reveal}{} or \Oracle{chall}{} per epoch.
Besides these constraints, the adversary is free to query \CGKAsec\ oracles and group members are free to call \CGKAdef\ algorithms concurrently.
Figure \ref{fig:CGKA-informal} depicts the complexities of this unconstrained, informal \CGKAsec, notably that the description yields a massively concurrent game, without a bounded win condition.
This is not always problematic for explicit state model checking, but regrettably it is with regards to \CGKAsec.
In order to reduce the complexity of the \CGKAsec, and efficiently produce an encoded model, formalization and subsequent abstraction of the \CGKAsec\ is required.

\begin{figure}[ht!]
\centering
\caption[Transition graph of informal CGKA definition]{%
\label{fig:CGKA-informal}%
High-level illustration of the informal CGKA definition's \Abrev{FSM}.%
Each depicted node represents a sub-component of the \Abrev{FSM}, with significant transitions displayed between the sub-components
}%
\resizebox{\textwidth}{!}{\subimport{../figures/}{CGKA-Informal}}
\end{figure}

\hypertarget{sec:game-prefixes}{%
\subsection{CGKA game prefixes}\label{sec:game-prefixes}}

Within the CGKA game the attacker makes a series of queries to the available oracles.
Queries to oracles drive the \Abrev{CGKA} protocol's state over time, however, when playing the game as originally defined there are no guaranteed limits on total attacker queries to all oracles.
Furthermore, while the attacker decides exactly which order to query which oracles, all group members of the \Abrev{CGKA} protocol concurrently broadcast an unlimited number of messages, both informational and control.
A direct modeling of the CGKA game with a \((T, C, N)\)-attacker would require \(N+1\) concurrent processes, accounting for the attacker and the maximum possible group members, along with \(N\) infinite queues of messages to be received by participants, and crucially monotonically increasing epoch and message counters.
Recall from Section \ref{sec:message-layer-security} that the \Abrev{DS} is guaranteed to provide all group members with a consistent view of message ordering.
This guarantee imposes a total ordering on messages delivered by the \Abrev{DS}, with the monotonically increasing epoch and message counters as the ordering key.
Spin only supports message queues of fixed capacity, but even attempting to circumvent the problem by establishing large capacity bounds for the message queues, a problem emerges for attempting a direct model encoding.
A single state in the state-space would constitute the unique combination of all actor processes' internal states, all message queue states, and the global protocol state.
Unfortunately, because of the monotonically increasing counters, there can be no finite representation of the infinite game.

To construct a finite model of an unbounded game, the states of the model must not depend on any monotonically increasing temporal information.
Temporal information which converges to a limit in finitely many steps or converges to a to a finite cycle in finitely many steps can all be finitely modeled.
If a model's state depends on non-converging temporal information, it necessitates that the state-vector length and search-space are unbounded.
This follows from the following three facts.

\begin{enumerate}
\def\labelenumi{\arabic{enumi}.}
\item Each state in the state-space must have a unique state-vector encoding.
\item The model's state and the equivalent state-vector's encoding must include monotonically increasing temporal information.
\item In an unbounded game, there is no finite encoding of monotonically increasing temporal information.
\end{enumerate}

While explicit state model checking is sophisticated, it ultimate reduces to an exhaustive search which uses a myriad of techniques to cleverly reduce the search space.
There does not exist an general technique for reducing an unbounded, potentially infinite, search space to a finite one, as this would yield a solution to the halting problem.
State-space dependence on temporality is a long-known difficulty with explicit state model checking, with a venerable solution is the application of timed automata \autocite{alur1994theory}.
Spin however does not support timed automata, requiring a different solution.
Considerable thought was given towards identifying a specific unbounded-to-finite reduction for the \CGKAsec, but the author is unable to present such a technique in this work.
Instead, different techniques are presented to ``truncate'' an equivalent finite ``prefix'' of the \CGKAsec.
In terms of a \((T, C, N)\)-attacker \(\mathcal{A}\), verification is performed from epochs $0$ to \Tmax, truncating the \CGKAsec\ to all prefix sequences containing at most \Tmax\ epochs.

The model of \CGKAsec\ for the TreeKEM protocol described subsequently in this chapter, is be parameterized by $t \in \{ 0, \Tmax \}$.
Model parameterization by $t$ means that the \CGKAsec\ is played until epoch $t$ is reached, at which point play continues \emph{without} advancing to epoch $t+$, with the attacker eventually ending the game with a call to \Oracle{deliver}{}.
Verification for the finite \CGKAsec\ prefixes less than or equal to \Tmax\ requires $|\{ 0, \Tmax \}|$ separate verification runs, one for verifying the CGKA model parameterized by each possible $t$ value with $t \in \{ 0, \Tmax \}$.
While simple and effective, this approach does involve duplicate verification work, with the verification of the model for $t$ being an optimal substructure for verifying the model parameterized by $t+1$.
The work presented here does not attempt memoization of sub-problems, though this is theoretically and practically possible with additional engineering effort.
For example of potential memoization, note that verifying the \(t+1\) epoch case requires also verifying the \(t\) epoch case.
So a model \CGKAmod{P}{T}{N} could be encoded to verify all an \Abrev{LTL} property for all \(t \le T\) by considering the possibility that the adversary queries \Oracle{chall} in each epoch \(t\).
However, the model encoding produced for this thesis instead forbids the adversary from querying \Oracle{chall} in any epoch \(t < T - 1 \), and requires that the adversary query \Oracle{chall} in epoch \(t = T\).
The encoding used, while simpler in construction, requires separate verification runs for all parameters \(T \in \{ 0, \Tmax \}\), rather than a single verification run with the parameter \Tmax.


\hypertarget{sec:exhaustiveness}{%
\subsection{Exhaustiveness}\label{sec:exhaustiveness}}

Verification through explicit state model checking relies on exhaustiveness as its proof method.
However, the definition of the \CGKAsec\ had no notion of considering every possible interleaving of actions when created.
Instead, it was defined in a manner which made existential proofs easy to describe and reason about.
Proofs and counterexamples from previous works \autocite{alwen2019double, alwen2020security} take the form of scrutinizing the existence of a sequence of queries made by the attacker.
This existential focus permits leniency for redundancy within the \CGKAsec\ definition.
Conversely, modeling the \CGKAsec\ such that it is amenable to explicit state model checking demands absolute parsimony within definition.
Considering the CGKA games definition through the lens of an exhaustive state-space search, both conveniently and surprisingly, leads to many possible game simplifications.
Because all states are considered, equivalent or superfluous sequences can be reasoned about, identified, and collapsed.

Consider the parameter \(C\) for a \((T, C, N)\)-attacker \(\mathcal{A}\) and recall from Section \ref{sec:game-adaptations} the constraints on \(\mathcal{A}\) regarding querying \Oracle{reveal}{} and \Oracle{chall}{}.
As defined, in each epoch \(\mathcal{A}\) must perform exactly one of the following:

\begin{itemize}
  \item Query \Oracle{reveal}{} exactly once.
  \item Query \Oracle{chall}{}  exactly once.
  \item Query neither \Oracle{reveal}{} or \Oracle{chall}{}.
\end{itemize}

Additionally, to terminate the \CGKAsec, \(\mathcal{A}\) must query \Oracle{chall}{} once.
Therefore, for all \((T, C, N)\)-attacker \(\mathcal{A}\), \(C \in [1, T]\).
Because the exhaustive state space of is explored, at each epoch the attacker chooses one of the three explicit querying options related to \Oracle{reveal}{} and \Oracle{chall}{}.
Hence within the model \CGKAmod{P}{T}{N}, the parameter \(C\) is linearly dependent on \(T\) for all \((T, C, N)\)-attackers and does not need to be a model parameter.
Similar reductions will follow while formalizing \CGKAmod{P}{T}{N} from the informal \CGKAsec\ definition.


\hypertarget{sec:idempotence}{%
\subsection{Idempotence}\label{sec:idempotence}}

A literal interpretation of the informal \CGKAsec\ definition permits many redundant state transitions.
Redundant in this context means the formation of cycles or ``non-progressing'' states.
For example, Suppose \(\mathcal{A}\) in state \(s_i\) queries \Oracle{no-del}{ID} transitioning to state \(s_{i+1}\), and then immediately performs an identical query to \Oracle{no-del}{ID} transitioning to \(s_{i+2}\).
This sequence of queries is not forbidden by the informal description of \CGKAsec.
However, it is apparent that the protocol states \(s_{i+1} = s_{i+2}\) as the second, identical query has not changed the protocol state.
Furthermore, if there were intervening queries made between states \(s_{i+1}\) and \(s_{i+2}\), it would still be the case that \(s_{i+1} = s_{i+2}\).
The effect these redundant transitions have on the model state is intuitively similar to an idempotent operation \(\star\), where \(a \star b \star b = a \star b\).
Hence an ``idempotency-like'' abstraction is applied to the model encoding forbidding the follow operations:

\begin{equation} \label{eq:7}
  \begin{split}
&\text{Let } \mathcal{Q} \text{ be a sequence of query and model state pairs.}\\
&\text{Let } (q_i, s_i) \text{ be the pair at index } i \text{ of } \mathcal{Q}.\\
&\text{Then forbid all } q_j \text{ such that:}\\
&\quad\quad\quad  (q_i = q_j \land i < j) \implies s_j = s_{j-1}
\end{split}
\end{equation}

More examples of permitted idempotency by the informal \CGKAsec\ definition (for some \(i\), and some \(t\)) are listed below:

\begin{itemize}
  \item \(\mathcal{A}\) queries \Oracle{reveal}{t}         two or more times throughout the \CGKAsec.  
  \item \(\mathcal{A}\) queries \Oracle{no-del}{ID_i}      two or more times throughout the \CGKAsec.
  \item \(\mathcal{A}\) queries \Oracle{corr}{ID_i}        two or more times during epoch \(t\).
  \item \(\mathcal{A}\) queries \Oracle{add-user}{ID_i}    two or more times during epoch \(t\).
  \item \(\mathcal{A}\) queries \Oracle{remove-user}{ID_i} two or more times during epoch \(t\).
  \item \(\mathcal{A}\) queries \Oracle{send-update}{ID_i} two or more times during epoch \(t\).
\end{itemize}

Since the verification methodology for \CGKAmod{P}{T}{N} exhaustively explores all possible \CGKAsec\ states, all sequences which includes an ``idempotency-like'' query, and all sequences which exclude the same query are equivalent computations within the encoded model.
Hence we can collapse the cyclical search through the countably infinite number of sequences including ``idempotency-like'' queries, considering them all to be the same unique state without effecting the verification of an LTL property \(\varphi\).
To disallow the cycles formed by idempotent operations, each operation above is permitted only once per period within which the operation is idempotent.
Within the \CGKAmod{}{}{}\ Promela encoding, repeated sequences of idempotent operations are tabulated and guarded against.
This has two important effects.

While these ``idempotency-like'' transitions can be detected and collapsed by Spin, but explicitly forbidding them allows removing some variables from the global state of the model.
With less information stored in the global state, each state-vector can be encoded using fewer bits.
The smaller state-vectors both reduce memory usage and improve verification run-time, though the former more than the latter.

Additionally, by disallowing idempotency within the game, each query made by the adversary will never result in a return to a previous game state.
A consequence of this, is that adversary queries necessarily advance the game state towards the next epoch transition, or when within the final epoch, advance towards the query to \Oracle{chall}{}.
This necessarily progression of the \CGKAsec\ within the model \CGKAmod{P}{T}{N} both reduces the run-time verification as well as permits verification that the model \CGKAmod{P}{T}{N} halts.


\hypertarget{sec:elided-group-members}{%
\subsection{Elided Group Members}\label{sec:elided-group-members}}

The complexity of the asynchronous, multi-party communication described by the informal \CGKAsec\ definition is illustrated within Figure\ \ref{fig:CGKA-informal}.
However, exhaustiveness can once again reduce model complexity.
Consider a \((T, C, N)\)-attacker \(\mathcal{A}\).
As informally defined in \autocite{alwen2020security}, in each epoch $t < T$ the transition from epoch $t$ to epoch $t+1$ can only occur in one of the following ways:

\begin{itemize}
  \item For some \(i\),\\~\hspace{4em}\(\mathtt{ID_i}\) calls \Protocol{add} and subsequently \(\mathcal{A}\) queries \Oracle{deliver}{t,\, ID_i,\, ID_{*} }
  \item For some \(i\),\\~\hspace{4em}\(\mathtt{ID_i}\) calls \Protocol{rmv} and subsequently \(\mathcal{A}\) queries \Oracle{deliver}{t,\, ID_i,\, ID_{*} }
  \item For some \(i\),\\~\hspace{4em}\(\mathtt{ID_i}\) calls \Protocol{upd} and subsequently \(\mathcal{A}\) queries \Oracle{deliver}{t,\, ID_i,\, ID_{*} }
  \item For some \(i\),\\~\hspace{4em}\(\mathcal{A}\) queries \Oracle{add-user}{ID_i}    and subsequently queries \Oracle{deliver}{t,\, ID_i,\, ID_{*}}
  \item For some \(i\),\\~\hspace{4em}\(\mathcal{A}\) queries \Oracle{remove-user}{ID_i} and subsequently queries \Oracle{deliver}{t,\, ID_i,\, ID_{*}}
  \item For some \(i\),\\~\hspace{4em}\(\mathcal{A}\) queries \Oracle{send-update}{ID_i} and subsequently queries \Oracle{deliver}{t,\, ID_i,\, ID_{*}}
\end{itemize}

While applying the ``idempotency-like'' abstraction described in \ref{sec:idempotence}, each of the six possible epoch initiation can occur independently in each epoch.
Prior to the next epoch initiation message being processed by the entire group, any possible combination of the six possibilities above can accumulate as options for \(\mathcal{A}\) to eventually select.
A consequence of nondeterministic computation, is that for an epoch containing \(n\) group members, each of the \(2^{6*n} - 1\) epoch transitions must be considered.
Only one epoch initiation message can be selected by \(\mathcal{A}\) via a query to \Oracle{deliver}{}, but the combinatorics of accumulated messages prior to selection still influence the search-space and state-vectors.
Removing this baggage on the state-vector encoding, similar to the ``idempotency-like'' abstraction, improving both memory usage and verification run-time.

The abstraction to achieve further model efficiency and scalability relies on the ability of \(\mathcal{A}\) to query \Oracle{deliver}{}.
In each epoch, \(\mathcal{A}\) decides which control message is delivered defining the next epoch.
Furthermore, note that \(\mathcal{A}\) querying \Oracle{add-user}{ID_i}, \Oracle{remove-user}{ID_i}, or \Oracle{send-update}{ID_i} results in the same control message being sent to the \Abrev{DS} as \(\mathtt{ID_i}\) calling \Protocol{add}, \Protocol{rmv}, or \Protocol{upd}; respectively.
Both the sequences originating with either \(\mathcal{A}\) querying an oracle or \(\mathtt{ID_i}\) calling an algorithm result in the same epoch transition once \Oracle{deliver}{t,\, ID_i,\, ID_{*}} is queried, without any other effect on the protocol state.
Due to the exhaustively methodology in verifying \CGKAmod{P}{T}{N}, both initiating epoch transitions options are explored, resulting in duplicate paths to the same epoch transition state.
Hence, we can omit the group members from calling calling \Protocol{add}, \Protocol{rmv}, or \Protocol{upd}, and without loss of generality, have \(\mathcal{A}\) query \Oracle{add-user}{ID_i}, \Oracle{remove-user}{ID_i}, or \Oracle{send-update}{ID_i}; respectively.
This final abstraction further removes global state variables from the model \CGKAmod{P}{T}{N}, the impact of which is reducing the memory usage by two orders of magnitude.


We arrive at a formalized and reduced \CGKAmod{P}{T}{N} model for a$( \textbf{T}, \textbf{C}, \textbf{N} )$-attacker depicted in Figure\ \ref{fig:CGKA-old-flowchart}.
Within the depiction, yellow nodes make nondeterministic choices, and dashed lines represent a nondeterministic path chosen by the origin node.
\begin{figure}
\centering
\caption[Flowchart of formalized CGKA security game.]{%
\label{fig:CGKA-old-flowchart}%
Flowchart of formalized CGKA security game.\\%
``Advances Epoch'' options involve querying \Oracle{add-user}{}, \Oracle{remove-user}{}, or \Oracle{send-update}{}\\%
``Gains Information'' options involve querying \Oracle{corr}{}, \Oracle{no-del}{}, or \Oracle{reveal}{}%
}
\resizebox{\textwidth}{!}{\subimport{../figures/}{flow-chart}}
\end{figure}

%\hypertarget{decidability}{%
%\subsection{Decidability}\label{decidability}}

%TODO:

%Bounded epochs to \(T\).
%No idempotent operations, each operation advances protocol state towards transition.
