%------------------------------------- Thesis Metadata -------------------------
%
% Thesis Information
\title%
           {Formal Verification Applications for the TreeKEM Continuous Group Key Agreement Protocol}
\date%
           {2022--07--15}
\annum%
           {2022}
\department%
           {Computer Science}

% Thesis Participants
\author%
           {Alex J. Washburn}
\advisor%
           {Subash Shankar}
\reader%
           {Sven Dietrich}
\director%
           {Saptarshi Debroy}

% Thesis Special Pages
\dedication%
{
This work is dedicated to future generations, with the hope that they experience secure communication which is intuitively usable, inveterately ubiquitous, and indelibly unrestricted.
}

\acknowledge%
{
First and foremost, I would like to thank my advisor Subash Shankar for his diligent guidance throughout my masters program.
His numerous insights, astute inquiries, and supportive direction were paramount in completing this work.
Similarly, I would like to thank Sven Dietrich and Saptarshi Debroy for their participation in my thesis defense and their contributions towards strengthening my final manuscript.
Additionally I would like to thank William Sakas for his leadership as department chair, making possible the masters program under which this work was conducted.\\
This thesis would not be possible without the Science Computer Cluster Facility of the American Museum of Natural History.
The computational resources and support which they provided were instrumental in facilitating the many phases of model encoding and subsequent verification.\\
Finally, I want to acknowledge the unconditional support of my partner, Erilia Wu. 
They remain an unwavering constant of stability through my research, as well as treasured wellspring of counsel which ceaselessly manifests in a myriad of serendipitous forms.
}

\abstract%
{
The features of Secure Group Messaging, the security guarantees of Message Layer Security, and the TreeKEM protocol designed to satisfy these guarantees and features are explored.
A motivation and methodology for verification via explicit model checking is presented.
Subsequently, a translation of the TreeKEM protocol into a Promela reference model is described, examining the nuances explicit model checking brings.
Finally the results of the formal verification methods are discussed.
}

\keywords%
{
Cryptography,
Formal Verification,
Linear Temporal Logic,
Model Checking,
Oracles,
Promela,
Scalability,
Security,
Spin,
TreeKEM
}
%------------------------------------- Thesis Metadata -------------------------
