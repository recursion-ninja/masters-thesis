%------------------------------------- Pandoc Templating -----------------------
%
% Thesis Metadata
\title       {Formal Verification of TreeKEM}
\date        {2022-06-20}
\year        {2022}
\department  {Computer Science}

% Thesis Participants
\author      {Alex Washburn}
\advisor     {Subash Shankar}
\reader      {Sven Dietrich}
\director    {Saptarshi Debroy}

% Thesis Special Pages
\dedication {%
This work is dedicated to future generations, with the hope that they experiance secure communication which is intuitively usable, inveterately ubiquitous, and indelibly unrestricted.
}

\acknowledge{%
First and foremost, I would like to thank my advisor Subash Shankar for his dilligent guidance throughout my masters program.
His numerous insights, astute inquiries, and supportive direction were paramount in completing this work.
Similarly, I would like to thank Sven Dietrich and Saptarshi Debroy for their participation in my thesis defense and their contributions towards strengthing my final manuscript.
Additionally I would like to thank William Sakas for his leadering as department chair, making possible the masters program under which this work was conducted.
Finally, I want to acknowledge the support of my partner, Erilia Wu, which both remained an unwavering constant and manifested in a myriad of forms.
}

\abstract{%
The features of Secure Group Messaging, the security guarantees of Message Layer Security, and the TreeKEM protocol designed to satisfy these guarantees and features are explored.
A motivation and methodology for verification via explicit model checking is presented.
Subsequently, a translation of the TreeKEM protocol into a Promela reference model is describe, examining the nuances explicit model checking brings.
Finally the results of the formal verifcation methods are discussed.
}

\keywords{%
Formal Verification, Model Checking, Promela, Spin, TreeKEM
}
%------------------------------------- Pandoc Templating -----------------------

\subimport{./}{abreviations}

%------------------------------------- Custom definitions ----------------------
\usepackage{ntheorem}

\theoremstyle{break}
\newtheorem{definition}{Definition}

\theoremstyle{remark}
\newtheorem*{remark}{Remark}

\theoremstyle{nonumberbreak}
\theorembodyfont{\upshape}
\theoremseparator{:}
\newtheorem{LTL}{LTL Predicate}
%------------------------------------- Custom definitions ----------------------
