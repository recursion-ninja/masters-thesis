\usepackage{adjustbox}
\usepackage{amsmath}
\usepackage{amssymb}
\usepackage{booktabs}
\usepackage{etoolbox}
\usepackage{graphicx}
\usepackage{import}
\usepackage{pifont}
\usepackage{stmaryrd} % double square bracket / parens
\usepackage{xfrac} % for sfrac in \NA
\usepackage{xspace}

\subimport{./}{abreviations}
\subimport{./}{metadata}
\subimport{./}{shorthand}

%------------------------------------- TikZ figure styles ----------------------
\usepackage{tikz}
\usetikzlibrary{tikzmark}
\usetikzlibrary{shapes,arrows,positioning}

% Boundary / labeling nodes
\tikzstyle{boxed} = [rectangle, draw, dashed, minimum height=9.5cm, minimum width=16.75cm]

\tikzstyle{label} = [text width=9.5cm, text centered,minimum width=9.5cm, rotate=90 ]

% Actor nodes	
\tikzstyle{ident} = [circle, draw, fill=green!40, text width=2em, text centered, minimum height=2em, node distance=3cm]

\tikzstyle{vdots} = [text width=2em, text centered, minimum height=2em, node distance=3cm]

\tikzstyle{agent} = [ellipse, draw, fill=green!40, text width=6em, text centered, minimum height=2em, node distance=2cm]

% Game/protocol interfaces
\tikzstyle{oracl} = [rectangle, draw, fill=blue!15, text width=7.5em, text centered, minimum height=2em, node distance=3cm]

\tikzstyle{algor} = [diamond, draw, fill=blue!30, text width=3.75em, text badly centered, node distance=2.5cm, inner sep=0pt]

\tikzstyle{cntrl} = [regular polygon, regular polygon sides=6, draw, fill=red!35, text width=1.2cm, text centered, minimum height=1.4cm, node distance=1cm, inner sep=0.0625mm]

\tikzstyle{ended} = [regular polygon, regular polygon sides=8, draw, fill=red!35, text width=1.2cm, text centered, minimum height=1.4cm, node distance=1cm, inner sep=0.0625mm]

% Transitions
\tikzstyle{line} = [draw, very thick, color=black!50, -latex']

\tikzstyle{choice} = [draw, thick,dash pattern={on 7pt off 2pt on 1pt off 3pt}, color=black!50, -latex']

\tikzstyle{triangle} = [fill=black!50, regular polygon, regular polygon sides=3, shape border rotate=180]
%------------------------------------- TikZ figure styles ----------------------


%------------------------------------- Custom definitions ----------------------
\usepackage{ntheorem}

\theoremstyle{break}
\newtheorem{definition}{Definition}

\theoremstyle{remark}
\newtheorem*{remark}{Remark}

\theoremstyle{nonumberbreak}
\theorembodyfont{\upshape}
\theoremseparator{:}
\newtheorem{LTL}{LTL Predicate}
%------------------------------------- Custom definitions ----------------------
