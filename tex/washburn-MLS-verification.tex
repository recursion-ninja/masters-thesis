
%\documentclass[11pt,letterpaper]{article}
\documentclass[12pt,twocolumn]{article}

%%%%%%%%%%%%%%%%%%%%%%%%%%%%%%%%%%%%%%%%%%%%%%%%%%%%%%%%%%%%%%%%%%%%%%%%%
\pagestyle{plain}                                                      %%
%%%%%%%%%% EXACT 1in MARGINS %%%%%%%                                   %%
\setlength{\textwidth}{6.5in}     %%                                   %%
\setlength{\oddsidemargin}{0in}   %% (It is recommended that you       %%
\setlength{\evensidemargin}{0in}  %%  not change these parameters,     %%
\setlength{\textheight}{8.5in}    %%  at the risk of having your       %%
\setlength{\topmargin}{0in}       %%  proposal dismissed on the basis  %%
\setlength{\headheight}{0in}      %%  of incorrect formatting!!!)      %%
\setlength{\headsep}{.25in}         %%                                   %%
\setlength{\footskip}{.5in}       


\renewcommand{\thesection}{\Roman{section}} 

\usepackage{amsmath}
\usepackage{amssymb}
\usepackage{natbib}
\usepackage{sectsty}
\usepackage{titlesec}

%\typeout{}

%\bibliographystyle{cbe}
%\citestyle{aa}
%\usepackage{fancyhdr}

%\fancyfoot{}
%\lhead{\centering \large Alex Washburn, computer science (Cryptography)}
%\pagestyle{fancy}
\pagenumbering{gobble}

\sectionfont{\fontsize{13}{14}\selectfont\centering}
%\sectionfont{\centering}
\titlespacing*{\section}{1pt}{1ex}{1ex}
\titlespacing*{\subsection}{1pt}{2ex}{0ex}

%\newcommand{\required}[1]{\section*{\hfil #1\hfil}}
%\renewcommand{\refname}{\hfil References Cited\hfil}
\bibliographystyle{plain}

\begin{document}
	
	\author{}
	\date{}
	\title{\vspace{-8ex} \textbf{Verification of the Secure Group Messaging}\\ \Large Alex Washburn\vspace{-7ex}}
	%\setlength{\droptitle}{-10em}   % This is your set screw
	\thispagestyle{plain}

	\twocolumn[
	\begin{@twocolumnfalse}
		
		\maketitle
		
		\begin{abstract}
		  I present the features of Secure Group Messaging, the security guarantees of Message Layer Security, and the TreeKEM protocol designed to satisfy these guarantees and features.
		  Next, a motivation for formal verification is argued based on its technical and societal merits.
		  I show that the how TreeKEM can be formally verified using explicit state model checking and provide a reference modeling of TreeKEM in Promela and verified by SPINS.
		\end{abstract}
	
	\end{@twocolumnfalse}
	]
	
	\section*{Introduction}
	
	In an ideal world, secure communication channels are available and accessible to members of our society.
	Secure communication facilitates open and confident communication between parties without fear of impersonation or message interception by malicious actors.
	Not only is secure communication important for individual use, but it is the cornerstone of commercial confidence in our progressively digitized economy.
	Without secure communication channels, online retail, banking, and collaboration would not exist as we know it today.
	The field of cryptography is the practice and study of techniques which lay the mathematical foundations for the secure communication we all benefit from daily.
	
	In the last two decades, the field of cryptography has made great progress in improving accessibility and usability of strong encryption.
	Secure, accessible, and intuitive communication between two parties is ubiquitous.
	Such secure bidirectional communication channels are known as Secure Messaging (SM).
	Secure Messaging is supported by a variety of services including ChatSecure, CryptoCat, Cyber Dust, Cyphr, Facebook Messenger, G Data Secure Chat, Gajim, GNOME Fractal, Google Allo, Haven, Kakao Talk, Line, Element, Signal, Silence, Silent Phone, Skype, Telegram, Viber, WhatsApp, and Wickr, Wire.
	However, generalizing the security guarantees of SM to communication with more than two parties is still an area of active research.
	This generalization to multi-party communication is known as Secure Group Messaging (SGM).
	
	Message Layer Security (MLS) \cite{Omara2020} is a framework for providing secure messages in groups of size two or more parties.
	The Internet Engineering Task Force (IETF) defines the MLS as an SGM protocol which provides a set of security guarantees that participants in such a protocol can expect.
	These security guarantees include end-to-end encryption \cite{padlipsky1978limitations}, message confidentiality, message integrity \cite{voydock1983security} and authentication \cite{jueneman1983message}, membership authentication \cite{chaum1985showing}, asynchronicity, forward secrecy \cite{gunther1989identity}, post-compromise security \cite{cohn2016post}, and scalability.
	Implementations conforming to the definition of MLS are a subset of possible SGM implementations.
	
	TreeKEM \cite{bhargavan:hal-02425247} is an proposed protocol which meets the definition of MLS.
	One of the novel features of SGM is the ability for participants to be added and removed from the secure communication channel.
	In the simpler case when two parties use SM, additional participants cannot be added. Rather than ``removing'' a participant, the communication channel is simply closed when one party wishes to cease communication.
	The SGM requirement of adding and removing participants to the secure communication channel poses interesting open challenges in the field of cryptography.
	TreeKEM specifies a cryptographic protocol for adding and removing participants while maintaining the security guarantees of MLS.
	However, this protocol has yet to be verified.
	
	The presence or absence of attacks on cryptographic protocols which violate one or more of the security guarantees are exceedingly difficult for humans to catch before widespread adoption \cite{clark1997survey}.
	Often attacks are discovered years, if not decades, after a protocol has been in active use.
	Late discovery of these attacks after widespread adoption of the protocol requires rapid effort to fix the protocol and encourage all systems using the protocol to update to the revised version.
	Often this updating process itself takes years of migration work or the migration never occurs.
	This is especially problematic for embedded systems which cannot be easily retrieved and modified.
	For these reasons, it is of paramount importance to identify all possible attacks on a cryptographic protocol as early as possible, so it can be corrected before widespread adoption.
	
	
	\section*{Proposal}
	
	I propose a project to formally verify the novel portion of the TreeKEM protocol which adds and removes group members.
	The results of such a verification will either produce a proof that the security guarantees cannot be violated under the protocol, or a cryptographic attack comprised of series of operations which violate one or more security guarantees.
	Verification providing a proof of security will help drive adoption of this protocol.
	In the latter scenario, I will then work with leading cryptographers to correct the protocol to resist the attack(s) discovered by the verification project.
	By verifying the protocol before widespread adoption, we have the opportunity to prevent potential future lapses in security, in the likely event that attacks are found later yet security corrections cannot be rapidly and uniformly applied.
	Either outcome of the verification project will advance the field of cryptography and improve secure communication channels for everyone.
	
	\section*{Justification}
	
	Why the model is accurate representation of CGKA game.
	
	\section*{Methodology}
	
	Formal verification takes many forms, with many techniques and tools at our disposal, and choosing the correct verification tool is of utter importance.
	In the past, Burrows–Abadi–Needham (BAN) \cite{burrows1989logic} logic was the tool of choice for verifying cryptographic protocols.
	Unfortunately, protocols verified by BAN logic were later found to have undiscovered attacks \cite{10.5555/188307.188350}.
	BAN was subsequently critiqued for not being able to fully verify crucial aspects of cryptographic protocols.
	Because of such concerns, results and conclusions from verification techniques using BAN logic stating that no attacks exist can be a false-positive.
	The kind false-positive verification of security that BAN logic provided could cause the very lapses in post-adoption security that the verification had attempted to preempt.
	BAN logic has since fallen out of favor and instead been replaced by model checking techniques \cite{marrero1997model}, which have been able to produce attacks on cryptographic protocols which BAN techniques incorrectly verified as secure.
	
	
	Model checking exists as the preferred alternative to BAN logic techniques for verifying cryptographic protocols \cite{kacprzak2006comparing} \cite{lomuscio2007verification} \cite{van2004symbolic} . Model checking represents all possible states of protocol as a finite diagram of transitions \cite{clarke1981design}. The model checker then asserts that one or more properties holds for the protocol by traversing through all combinations of transitions through the diagram. If a property holds, the model checker will assert that there is no possible set of transitions through states that exist in the protocol which violate the specified property. If the property does not hold, the model checker will produce a counter-example sequence of transitions through possible states showing how the specified property can be violated. This method of verifying cryptographic protocols is very attractive as the result is either an exhaustive proof of security guarantees or a novel cryptographic attack which can be analyzed and corrected. 
	
	Correctly specifying the properties which must hold for the cryptographic protocol is one of the most important aspects of model checking techniques. Fortunately, Linear Temporal Logic (LTL) \cite{4567924} is an almost perfectly tailored language for representing the security guarantees of a cryptographic protocol. LTL allows a precisely definition of what must occur in specific segments of time during the the protocol's transition through possible states. An example of LTL's expressive power is that it allows for stating that a property must always hold, eventually hold, hold until another property holds, holds in the next state, never holds, or any combination of these. LTL possesses just enough expressive power to specify the requirements of the cryptograpic protocol's security guarantees, but no more than that. This perfect balance of expressiveness is important, because the more expressive the logical framework we choose to define our security guarantees, the less efficient the available verification techniques will be.
	
	Once we decided to use model checking with LTL as the formal verification methodology, we must then select the appropriate tools to use - the Protocol Meta Language (PROMELA) \cite{holzmann1990design}. It exists as a precise and convenient specification for defining all possible states and transitions for TreeKEM. By specifying the protocol in PROMELA, we can allow the PROMELA ``translator'' to take a high-level, human readable description of TreeKEM and produce a much larger machine readable list with all the possible states and transitions in TreeKEM. Manually defining all these states and transitions would take an inordinate amount of time and is prone to errors. Producing an exhaustive and correct representation of TreeKEM's possible states and transitions is one aspect of the verification process, at which machines are exceptionally reliable, while humans are not.
	
	Once PROMELA is used to produce all possible states and transitions of TreeKEM, we must supply that representation to a model checker which will verify LTL specifications. Among the many options of model checkers, we will use the SPIN model checker \cite{holzmann1997model}. SPIN provides several important features for our verification goals. The main difficulty with model checking is the size of the number of states and transitions \cite{burch1992symbolic}. Since model checking works by checking every possible combination of states and transitions, the size of sets of states directly impacts the feasibility. SPIN provides some solutions to making this problem tractable. First, SPIN supports ``on-the-fly'' verification \cite{couvreur1999fly}, which means that SPIN does not need to consider the entire set of states and transitions at the same time, but instead can work on just subsets until the whole set is verified. Second, SPIN uses a ``partial ordering'' of the states and transitions \cite{godefroid1990using} to reduce the number of subsets it needs to verify before being certain that a property holds for the whole set of states and transitions. Because of these (and many more) techniques, SPIN allows us to efficiently work with a large ``state-space'' that the novel portions of TreeKEM are likely to produce when specified with PROMELA.
	
	There is one last important detail the verification. Recall that what differentiates SM from SGM is that SGM involves two or more parties. There is no theoretical limit to the number of parties the protocol supports. However we also stated that model checking represents all possible states of protocol as a finite diagram of transitions. Here we have a potential problem, where TreeKEM allows an unbounded number of participants, producing an infinite number of possible states, while our model checker requires a finite number of states. The IETF states that MLS in practice should support approximately 50,000 participants \cite{Omara2020}, which is a finite, yet very large number that is problematic for model checking. To rectify this, we will use the mathematical principle of induction. Induction allows us to show that if the security guarantees hold for $n$ parties participating in the communication channel, that the security guarantees will also hold for $n+1$ parties, where $n$ has small and tractable lower bound. By applying induction, we only need to verify that the security guarantees hold up to the lower bound of $n$ parties (perhaps $n=5$), and that the inductive step also holds. By using the principle of induction, we can verify that the security guarantees hold for any number of participating parties.
	
	\section*{CGKA}
	
	Describe the oracle based security game
	
	Describe impotency of queries. Describe totality and terminate of game.
	
	\section*{Conclusion}
	
	Our goal is to use formal methods to verify the novel portions of the TreeKEM cryptograpic protocol before widespread adoption to mitigate potential disastrous future security lapses. To do so, we will translate the security guarantees defined by MLS into LTL, and use PROMELA to produce a model of the novel portions of TreeKEM. Lastly, we will use SPIN to produce either an exhaustive proof or counter example cryptographic attack(s) for each security guarantee of the protocol. The techniques described for completing this project represent a feasible and well defined plan. Furthermore, this project constitutes a productive effort towards improving the accessibility and confidence in secure communication channels available to the public. I am confident in this project's successful completion and positive broader impact.
	
	\onecolumn
	\bibliography{thesis-refs}
	
\end{document}